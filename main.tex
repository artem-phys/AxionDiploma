\documentclass[a4paper,article,14pt]{extarticle}

\usepackage{audiploma}
\usepackage{euscript}
\usepackage{longtable}
\usepackage{makecell}
\usepackage[pdftex]{graphicx}
\usepackage{amsthm,amssymb, amsmath}
\usepackage{textcomp}


\begin{document}

% --------------------- Стандарт СПб АУ РАН --------------------------%

\begin{titlepage}

\newgeometry{left=30mm, top=30mm, right=15mm, bottom=30mm, nohead, nofoot}

\vspace{15mm}
\begin{center}
    \includegraphics[width = \textwidth]{images/autitle.png}
\end{center}

\vspace{0.1mm}

\begin{flushright}
    «\rule{1cm}{0.15mm}» \rule{2cm}{0.15mm} 2021г. \\
    Зав. каф. Общей и \\
    теоретической физики \\
    \rule{25mm}{0.20mm} д.ф.-м.н. С. А. Тарасенко \\
\end{flushright}


\begin{center}

\vspace{9mm}
\textbf{\large ИСПОЛЬЗОВАНИЕ ТУЛИЕВЫХ БОЛОМЕТРОВ В КАЧЕСТВЕ ПЕРЕСПЕКТИВНЫХ ДЕТЕКТОРОВ СОЛНЕЧНЫХ АКСИОНОВ}
\vspace{9mm}

выпускная квалификационная работа бакалавра \\
\vspace{10mm}
Направление 03.03.01 Прикладные математика и физика \\
\vspace{14mm}
\textbf{\large Кузьмичев Артем Михайлович}


\vspace{16mm}

% Научный руководитель
\textbf{Научный руководитель} \hfill \rule{6.5cm}{0.15mm} Е.В. Унжаков \\
\vspace{5mm}
\textbf{Студент} \hfill  \rule{6cm}{0.15mm} А.М. Кузьмичев

\vfill 

{Санкт-Петербург, 2021}
\end{center}
\end{titlepage}
% Возвращаем настройки geometry обратно (то, что объявлено в преамбуле)
\restoregeometry
% Добавляем 1 к счетчику страниц ПОСЛЕ titlepage, чтобы исключить
% влияние titlepage environment
\addtocounter{page}{1}
\tableofcontents
\pagebreak

\specialsection{Введение}

Стандартная модель в настоящее время является наиболее успешной теорией, описывающей элементарные частицы и их взаимодействия. Тем не менее, существует целый ряд наблюдений и экспериментов, для которых Стандартная модель не даёт адекватного объяснений.

Появление в теории гипотетической псевдоскалярной частицы - аксиона - связано с проблемой ненаблюдения нарушения CP-симметрии в сильных взаимодействиях. Так называемый $\theta$-член в лагранжиане квантовой хромодинамики (КХД) отвечает за взаимодействие глюонных полей и имеет следующий вид:



Тулий-169 имеет низколежащий ядерный уровень 8.41 кэВ, что даёт возможность взять его как ядро-мишень для поиска резонансного поглощения солнечных аксионов. Планируется использование тулийсодержащего кристалла семейства гранатов $Tm_3Al_5O_{12}$ в качестве болометрического детектора . 

С данной целью был выращен образец кристалла и испытаны его болометрические и оптические свойства. Измерены химические и радиоактивные загрязнения, произведён расчёт с учётом эффективности регистрации детектора, оценка которой производилась методом Монте-Карло в Geant4

В данной работе представлен общий обзор проблематики поиска солнечных аксионов, результаты текущих исследований и оцениваем требования к будущей низкофоновой экспериментальной установке.

\specialsection{Постановка задачи}




\specialsection{Обзор экспериментов по поиску аксиона}

Есть разные эксперименты

\subsection{Постановка задачи}

Поставть предел на массу аксиона

\section{Основная часть раз}
Лагранжиан, описывающий взаимодействие аксионного поля $\phi_A$ с электромагнитным полем, задаваемым тензором $F^{\mu \nu}$:
\begin{equation}
    \mathcal{L}  = {g_{A\gamma }}{\varphi _A}{\varepsilon _{\alpha \beta \mu \nu }}{F_{\alpha \beta }}{F^{\mu \nu }} = {g_{A\gamma }}{\varphi _A}\vec B \cdot \vec E
\end{equation}
Константа свзяи с фотоном $g_{A\gamma}$ в моделях "невидимого аксиона равна:
\begin{equation}
   {g_{A\gamma }} = \frac{\alpha }{{2\pi {f_A}}}\left[ {\frac{E}{N} - \frac{{2\left( {4 + z} \right)}}{{3\left( {1 + z} \right)}}} \right] = \frac{\alpha }{{2\pi {f_A}}}{C_{A\gamma \gamma }}
\end{equation}

Данное взаимодействие ответственно за рождение аксионов на Солнце вследствие конверсии фотонов в электромагнитном поле. Для аксионов, достигающих поверхность Земли, энергетический спектр определяется следующим выражением[ссылька1, ссылка2]:
dphi/dea

\subsection{Резонансное поглощение в ядерных переходах магнитного типа}

***

Скорость поглощения солнечных аксионов ядром $^{169}Tm$ в единицу времени $R_A$, измеряемая в $\left( atom^{-1} \cdot c^{-1}\right)$ составит:
\begin{enumerate}
    \item[•] в терминах констант связи
    \begin{equation}
        {R_A} = {C_{Ax}} \cdot g_{Ax}^2{\left( {g_{AN}^0 + g_{AN}^3} \right)^2}{\left( {\frac{{{p_A}}}{{{p_\gamma }}}} \right)^3}
    \end{equation}
    \item[•] в терминах массы аксиона
    \begin{equation}
   {R_A} = {C''_{Ax}}m_A^4{\left( {\frac{{{p_A}}}{{{p_\gamma }}}} \right)^3}
\end{equation}
\end{enumerate}


Константы $C_{Ax }$, зависящие от аксионной модели, мишени и др. параметров, были вычислены для ядер $^{169}Tm$ в работах [14,30]: $C_{A\gamma } = 104 $ $C_{Ae} = 2.76 \cdot {10^5}$. Соотве




где $m_A$ - масса аксиона в эВ.



Полное число зарегистрированных событий в пике, который можно сопоставить с аксионом, пропорционально числу ядер $^{169}Tm$ в мишени, времени измерений и эффективности регистрации детектором. 
Вероятность зарегистрировать аксионный пик зависит от уровня фона и разрешения детектора

Найдём число ядер в мишени ${N_{Tm}}$

Вычислим молярную массу вещества детектора:
\begin{equation}
    \mu \left( {T{m_3}A{l_5}{O_{12}}} \right) = 3 \cdot 168.93 + 5 \cdot 26.98 + 12 \cdot 16 = 833.69\frac{g}{{mol}}
\end{equation}

Каждая молекула мишени содержит 3 искомых ядра, поэтому

\begin{equation}
    N_{Tm} = 3\nu  \cdot {N_A} = 3\frac{m}{\mu }{N_A} = 3\frac{m}{\mu }{N_A}
\end{equation}

Подставляя $m=
8.18$, получаем ${N_{Tm}} = 3 \cdot \frac{{8.18}}{{833.69}} \cdot 6.022 \cdot {10^{23}} \approx 1.77 \cdot {10^{22}}$



Полагая:
\begin{itemize}
    \item $N_{Tm} = 1.77 \cdot {10^{22}}$
    \item $\varepsilon \sim 1 $, так как в болометрических детекторах ядра мишени находится непосредственно внутри активного объема
    \item Время экспозиции 1 год: $T = 3.15 \cdot {10^7} c$
\end{itemize}

Мы можем записать предел:
\begin{equation}
   \varepsilon  \cdot T \cdot {R_A} \cdot N_{Tm} \leqslant {S_{\lim }}
\end{equation}


\newpage

\specialsection{Выводы}
Проведённые расчёты показывают, что создание криогенной установки на основе тулиевых болометров может улучшить существующие экспериментальные пределы на содержание Америция на несколько порядков
\pagebreak

\specialsection{Заключение}
Основные результаты, полученные в работе, заключаются в
следующем:
1. Создана экспериментальная установка с Si(Li)-детекторами и
мишенью из 169Tm. Низкофоновая установка включает в себя пассивную и
активную защиту от космического излучения, а также регистрирующую
аппаратуру.
2. Создана программа накопления данных с Si(Li)-детекторов,
позволяющая проводить длительные измерения и контролирующая работу
детекторов и активной защиты. Создана программа для расчета
эффективности регистрации гамма-квантов для различной геометрии между
планарным детекторам и мишенью.
3. Проведен поиск резонансного поглощения солнечных аксионов,
возникающих в результате конверсии тепловых фотонов в поле плазмы,
ядрами 169Tm, приводящего к возбуждению первого ядерного уровня  соответствующий энергии первого
возбужденного уровня 169Tm, статистически не проявился, что позволило

Полученные результаты были представлены на 146 международной
конференции по проблемам ядерной спектроскопии и структуре атомного
ядра (Кибердянск 2077) и опубликованы в работах



% Библиография в cpsconf стиле
% Аргумент {1} ниже включает переопределенный стиль с выравниванием слева
\begin{thebibliography}{1}
\bibitem{voc} Griffin D.W., Lim J.S. \flqq Multiband excitation vocoder\frqq. IEEE ASSP-36 (8), 1988, pp. 1223-1235.
\bibitem{vo2} Griffin D.W., Lim J.S. \flqq Multiband excitation vocoder\frqq. IEEE ASSP-36 (8), 1988, pp. 1223-1235.
\end{thebibliography}
\end{document}